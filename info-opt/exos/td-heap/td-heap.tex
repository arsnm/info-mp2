\documentclass{article}

% PACKAGES
% essentials
\usepackage[french, english]{babel} % specific language formatting
\usepackage[T1]{fontenc} % characters encoding
\usepackage{caption} % captions for graphics, images, ...

% page formatting
\usepackage[a4paper, portrait, margin=20mm]{geometry} % define the page format
\usepackage[explicit]{titlesec} % formatting and styling of section titles ... 
% ... and headers
\usepackage{fancyhdr} % customize headers and footers
\usepackage{lastpage} % keeps track of the number of last page
\usepackage{graphicx} % image, figures and graphics insertion
\usepackage[ddmmyyyy]{datetime} % display the date and time of saving
\usepackage{adjustbox} % scaling, resizing graphics
\usepackage{color, soul} % text formatting (highlight, underline, ...)
\setulcolor{red}
\usepackage{enumitem, hyperref}

% math notation
\usepackage{amsmath, amsfonts, amssymb}
\usepackage{stmaryrd}

% code formatting
\usepackage[outputdir=.aux]{minted}
\usemintedstyle{emacs}

%------------------------------------------------------------------------------

% COMMANDS AND PERSONALIZATION

% personalized section style
\titleformat{\section}
{\Large\bfseries}
{\thesection}{1em}{\ul{#1}}

% personalized 'question' list (mainly for test sheet)
\newlist{question}{enumerate}{2}
\setlist[question, 1]
{label=\bfseries{\arabic{questioni}.},
leftmargin=5em,
% resume
}
\setlist[question, 2]
{label=(\alph{questionii}),
leftmargin=\parindent
}

% math commands
\newcommand{\bb}[1]{\mathbb{#1}}

% framing text
\newcommand{\encadrer}[1]{\fbox{color=red
    \begin{minipage}{0.90\textwidth}
        #1
    \end{minipage}
}}

\renewcommand{\thesection}{\Roman{section}} % Roman numerals for sections
%\renewcommand{\thesubsection}{\thesection.\Alph{subsection} -} %  Alphabetical
% numerals for subsections
\setlength{\headheight}{12.5pt}

\graphicspath{ {./img/} } % define path to img 

% insert image command
\newcommand{\image}[3]{
    \begin{minipage}[t]{\linewidth}
        #1
        \adjustbox{valign=t}{
            \includegraphics[width=#2\linewidth]{#3}
        }
    \end{minipage}}

% page numbering
\pagestyle{fancy}
\fancyhf{}
\renewcommand{\headrulewidth}{0pt}
\fancyfoot[R]{\thepage/\pageref{LastPage}}

%------------------------------------------------------------------------------

% DOCUMENT
% document info
\makeatletter
\title{TD - Tas et File de Priorité}
\date{\today}
\newcommand{\matiere}{Informatique Option}
\newcommand{\classe}{MP*}
\author{Arsène MALLET}

% header
\fancypagestyle{firstpage}{
    \fancyhead[L]{\@author}
    \fancyhead[C]{\classe - \matiere}
    \fancyhead[R]{\@date}
}


\begin{document}

% personalized list style


\thispagestyle{firstpage}

\begin{center}
    \huge\bfseries{\@title}
\end{center}

\section{Un tas de questions}


\section{Compression de Huffman}

\begin{question}

    \item \begin{minted}{ocaml}
let rec read tree list =
    match tree with 
    |F(a) -> a, list 
    |N(g, d) -> let h, t = List.hd list, List.tl list in
                if h = 0 then
                    read g t
                else if h = 1 then
                    read d t
                else failwith "Provided list does not correspond to a char"
    ;;
    \end{minted}

    \item \begin{minted}{ocaml}
let rec decode tree list =
    if list = [] then [] else 
    let char, remainder = read tree list in
        char::decode tree remainder 
    ;;
    \end{minted}

    \item 

    \item \begin{minted}
let
\end{minted}
\end{question}


\section{Arbretas}

\begin{question}

    \item \begin{minted}{ocaml}
let swap tableau i j =
    let temp = tableau.(i) in
    begin
    tableau.(i) <- tableau.(j);
    tableau.(j) <- temp ;
    end  ;;
\end{minted}

    \item TODO

    \item Montrons le par récurrence sur le nombre d'élément de l'arbre.
Notons $\mathcal{P}_k$, l'assertion "Pour $k$ couples distincts, il existe un unique
arbretas les contenants".

    \begin{itemize}
        \item Initialisation : si $k = 0$, alors l'arbre vide convient.

        \item Soit $k \in \bb{N*}$, supposons $\mathcal{P}_{k - 1}$:
            Soit $k$ couples distincts, on les note 
            $E = (e_1, p_1), \dots (e_k, p_k)$
            On suppose sans perdre de généralité que les $p_i$ sont triés dans l'ordre
            décroissant.
            
            Un arbretas contenant $E$ est donc de la forme $N(r, g, d)$ où $r = (e_k, p_k)$
            et $g, d$ sont des arbrestas contenant au plus $k - 1$ elements donc sont 
            uniques d'après $P_k$. 
    \end{itemize}
     D'où le résultat.

    \item \begin{minted}{ocaml}
let rotd abr = 
    match abr with
        |N(r, N(gr, gg, gd), d) -> N(gr, gg, N(r, gd, d))
        |_ -> abr
    \end{minted}

    \item TODO

    \item \begin{minted}{ocaml}
let prio = function
        |V -> max_int
        |N(r, g, d) -> r
    \end{minted}

    \item \begin{minted}{ocaml}
let rec add abr (element, priorite) = 
    match abr with
    |V -> N((element, priorite), V, V)
    |N((element_r, priorite_r), g, d) when element < element_r ->
        if priorite < priorite_r then 
            rotd (N((element_r, priorite_r), add g (element, priorite), d))
        else 
            N((element_r, priorite_r), add g (element, priorite), d)
    |N((element_r, priorite_r), g, d) ->
        if priorite < priorite_r then 
            rotg (N((element_r, priorite_r), g, add d (element, priorite)))
        else 
            N((element_r, priorite_r), g, add d (element, priorite))
;;
    \end{minted} 

    \item TODO \begin{minted}{ocaml}

    \end{minted} 
\end{question}
\end{document}
