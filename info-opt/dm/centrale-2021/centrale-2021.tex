\documentclass{article}

% PACKAGES
% essentials
\usepackage[french, english]{babel} % specific language formatting
\usepackage[T1]{fontenc} % characters encoding
\usepackage{caption} % captions for graphics, images, ...

% page formatting
\usepackage[a4paper, portrait, margin=20mm]{geometry} % define the page format
\usepackage[explicit]{titlesec} % formatting and styling of section titles ... 
% ... and headers
\usepackage{fancyhdr} % customize headers and footers
\usepackage{lastpage} % keeps track of the number of last page
\usepackage{graphicx} % image, figures and graphics insertion
\usepackage[ddmmyyyy]{datetime} % display the date and time of saving
\usepackage{adjustbox} % scaling, resizing graphics
\usepackage{color, soul} % text formatting (highlight, underline, ...)
\setulcolor{red}
\usepackage{enumitem, hyperref}

% math notation
\usepackage{amsmath, amsfonts, amssymb, mathrsfs}
\usepackage{stmaryrd}

% code formatting
\usepackage[outputdir=.aux]{minted}
\usemintedstyle{emacs}

%------------------------------------------------------------------------------

% COMMANDS AND PERSONALIZATION

% personalized section style
\titleformat{\section}
{\Large\bfseries}
{\thesection}{1em}{\ul{#1}}

% personalized 'question' list (mainly for test sheet)
\newlist{question}{enumerate}{2}
\setlist[question, 1]
{label=\bfseries{Q\arabic{questioni}.},
leftmargin=5em,
resume
}
\setlist[question, 2]
{label=(\alph{questionii}),
leftmargin=\parindent
}

% math commands
\newcommand{\bb}[1]{\mathbb{#1}}

% framing text
\newcommand{\encadrer}[1]{\framebox{color=red
    \begin{minipage}{0.90\textwidth}
        #1
    \end{minipage}
}}

\renewcommand{\thesection}{\Roman{section}} % Roman numerals for sections
\renewcommand{\thesubsection}{\thesection.\Alph{subsection} -} % Alphabetical
% numerals for subsections
\setlength{\headheight}{12.5pt}

\graphicspath{ {./img/} } % define path to img 

% insert image command
\newcommand{\image}[3]{
    \begin{minipage}[t]{\linewidth}
        #1
        \adjustbox{valign=t}{
            \includegraphics[width=#2\linewidth]{#3}
        }
    \end{minipage}}

% page numbering
\pagestyle{fancy}
\fancyhf{}
\renewcommand{\headrulewidth}{0pt}
\fancyfoot[R]{\thepage/\pageref{LastPage}}

%------------------------------------------------------------------------------

% DOCUMENT
% document info
\makeatletter
\title{Centrale 2021}
\date{\today}
\newcommand{\matiere}{Option Informatique}
\newcommand{\classe}{MP*}
\author{Arsene MALLET}

% header
\fancypagestyle{firstpage}{
    \fancyhead[L]{\@author}
    \fancyhead[C]{\classe - \matiere}
    \fancyhead[R]{\@date}
}


\begin{document}

\thispagestyle{firstpage}

\begin{center}
    \huge\bfseries{\@title}
\end{center}

\section{Quelques Fonctions Auxiliaires}

\begin{question}
    \item \begin{minted}[breaklines]{ocaml}
let nombre_aretes g =
    let rec length_list l =
        match l with
        |[] -> 0
        |h::t -> 1 + length_list t 
    in
    let s = ref 0 in
    for i = 0 to Array.length(g) - 1 do
        s := !s + length_list g.(i)
    done;
    !s / 2 ;;
    \end{minted}

    \item \begin{minted}[breaklines]{ocaml}
let g_2_3 = [|
    [|3; 1|];
    [|4; 0; 2|];
    [|5; 1|];
    [|0; 4|];
    [|1; 3 ; 5|];
    [|2; 4|];
    |];;
    \end{minted}

    \item \begin{minted}[breaklines]{ocaml}
let adjacence g =
    let n = Array.length g in
    let adj = Array.make n [||] in
    for i = 0 to n - 1 do
        adj.(i) <- Array.of_list g.(i)
    done;
    adj ;; 
    \end{minted}
    
    \item \begin{minted}[breaklines]{ocaml}
let rang (p, q) (s, t) =
    let is, js = s / p, s mod p in
    let it, jt = t / p, t mod p in
    if it = is + 1 then
        (q - 1) * js + is
    else if jt = js + 1 then
        p * (q - 1) + (p - 1) * is + js
    else 
        failwith "Argument(s) invalide(s)" ;;
    \end{minted}

    \item \begin{minted}[breaklines]{ocaml}
let sommets (p, q) rg =
    if rg < p * (q - 1) then
        let is, js = rg mod (q - 1), rg / (q - 1) in
        let s = is * p + js in
        (s, s + p)
    else if rg < p * (q - 1) + q * (p - 1) then
        let shift = p * (q - 1) in
        let is, js = (rg - shift) mod (p - 1), (rg - shift) / (p - 1) in
        let s = js * (q + 1) + is in
        (s, s + 1)
    else failwith "Argument(s) invalide(s)" ;;
    \end{minted}

    \item \begin{minted}[breaklines]{ocaml}
let quadrillage p q =
    let graphe = Array.make (p * q) [] in
    let rec remplissage_graphe rg =
        if rg < p * (q - 1) + q * (p - 1) then
            let v1, v2 = sommets (p, q) rg in
            begin
            graphe.(v1) <- v2 :: graphe.(v1) ;
            graphe.(v2) <- v1 :: graphe.(v2);
            remplissage_graphe (rg + 1);
            end
    in
    remplissage_graphe 0 ;
    graphe ;; 
    \end{minted}
\end{question}

\section{Caractérisation des arbres}

\subsection{Propriétés sur les arbres}

\begin{question}
    \item Si $s, t \in S_n$, notons $s \ast t$, la relation "Il existe un
    chemin de $s$ à $t$". \\
    Montrons que $\ast$ est une relation d'équivalence sur $S_n$.
    
    \begin{itemize}
        \item Réflexivité : soit $s \in S_n$. Par convention, il existe 
        un chemin de $s$ à $s$. Donc $s \ast s$.

        \item Symétrie : soit $s, t \in S_n$, si $s \ast t$, alors il existe 
        un chemin $c = (s, s1, \hdots, s_{k - 1}, t)$. 
        Donc $\forall i \in \{0, \hdots, k - 1\}$, $\{s_i, s_{i + 1}\} \in A$
        donc $\{s_{i + 1}, s_i\} \in A$, donc le chemin 
        $c' = (t, s_{k - 1}, \hdots, s_1, s)$ existe et donc $t \ast s$

        \item Transitivité : soit $s, t, u \in S_n$ tels que si $s \ast t$ et
        $t \ast u$. Alors il existe $c1 = (s, s_1, \hdots, s_{k - 1}, t)$ \\
        et $c2 = (t, t_1, \hdots, t_{i - 1}, u)$. \\
        Donc en concaténant ces chemins, il existe 
        $c = (s, \hdots, s_{k - 1}, t, t_1, \dot, t_{k - 1}, u)$, 
        d'où $s \ast u$.
    \end{itemize}

    Ainsi comme les composantes connexes de $G$ sont les classes d'équivalence
    de $\ast$, elles forment une partition de $S_n$.

    \item Soit $s, t$ deux sommets tels que $s \ast t$, en notant $len(c)$ la
    longueur d'un chemin $c$, \\ alors 
    $L = \{ len(c) \vert c \text{ chemin de $s$ à $t$}\}$ est une partie de
    $\bb{N}$, non-vide (puisque $s \ast t$), \\ donc il existe un plus petit 
    élément $k_0$ de $L$. D'où l'existence d'un plus court chemin de $s$ à $t$.

    Soit $c_0$ un plus court chemin de $s$ à $t$, notons le
    $c_0 = (s, s_1, \hdots, s_{k_0 - 1}, t)$. \\
    Si il existe $i \neq j$ tels que $s_i = s_j$ (on peut supposer sans perte
    de généralité que $i < j$) alors $c = (s, \hdots, s_i = s_j, s_{j + 1}, 
    \hdots, t)$, un chemin de longueur $k_0 - (j - i) < k_0$, ce qui contredit
    le caractère de plus court chemin de $c_0 \rightarrow$ absurde. Donc les
    sommets d'un plus court chemin sont distincts.

    \item \label{itm:Q9} Soit $k \in \llbracket 0, m \llbracket$, notons $s, t$ les extrémités
    de $a_k$.

    Supposons que $s$ et $t$ appartiennent à la même composante connexe de
    $G_k$, alors $s \ast_k t$. Ainsi en notant 
    $c_k = (s_0 = s, s_1, \hdots, s_{i - 1}, s_i = t)$ (avec $i > 1$), où les
    sommets de $c_k$ sont adjacents dans $G_k$, alors il existe un chemin $c$ 
    dans $G$ tel que $c = (s, \hdots, t, s)$ (car $a_k$ relie $s$ et $t$). Or
    $len(c) = i + 1 \geqslant 2$, donc il existe un cycle dans $G$. Or $G$ est 
    un arbre donc est acyclique $\rightarrow$ absurde !

    Ainsi les extrémités de $a_k$ appartiennent à deux composantes connexes
    différentes de $G_k$.

    En notant pout tout $i \in \llbracket 0, m \rrbracket$,
    $\varphi(i)$ le nombre de composantes connexes de $G_i$, alors
    $\phi(0) = n$ ($G_0$ est composé de $n$ sommets non reliés) et $\varphi(m) = 1$ ($G_m = G$ est
    un arbre, donc connexe).

    Donc si $k \in \llbracket 0, m \llbracket$ et $a_k = \{s, t\}$, alors
    d'après ce qu'on a fait juste avant, $s$ et $t$ sont dans deux composantes
    connexes différentes de $G_k$ et dans la même dans $G_{k + 1}$. Les
    composantes connexes étant disjointes, si $u \in S_n$ tel que $C_{u_k} \neq
    C_{s_k}$ et $C_{u_k} \neq C_{t_k}$, alors $C_{u_k} = C_{u_{k + 1}}$, 
    (les $C_{i_k}$ étant les composantes connexes de $G_k$ contenant $i$), d'où
    finalement $\varphi(k + 1) = \varphi(k) - 1$

    Par une récurrence immédiate, $\varphi(m) = \varphi(0) - m$, d'où 
    $m = n - 1$ et donc le résultat.

    \item D'après \ref{itm:Q9}, $(i) \implies (ii)$ et $(i) \implies (iii)$
    
    Si $(ii)$, notons 
    $\mathcal{H} = \{ H \mid H = (S_n, B), B \subset A \text{ et H connexe}\}$,

    Comme $G \in \mathcal{H}$, alors $\mathcal{H}$ non vide, en particulier,
    l'ensemble des cardinaux des aretes est une partie non-vide de $\bb{N}$, on 
    peut donc considérer $H = (S_n, B)$ avec $B$ de cardinal minimal. 

    Supposons par l'absurde que $H$ possède un cycle, alors en supprimant une
    arete quelconque de ce cycle, $H$ reste connexe et possède $|B| - 1 < |B|$
    aretes $\rightarrow$ absurde car $B$ est de cardinal minimal. Donc $H$ est
    acyclique et finalement $H$ est un arbre, d'où $|B| = n - 1$
    (d'après \ref{itm:Q9}) et donc finalement $H = G$.

    Ainsi $G$ est un arbre donc $(ii) \implies (i)$

    \vspace{1em}

    Si $(iii)$, notons de même
    $\mathcal{H} = \{H \mid H = (S_n, B),A \subset B \text{ et H acyclique}\}$,

    Comme $G \in \mathcal{H}$, alors $\mathcal{H}$ non vide, alors on peut
    de même considérer $H = (S_n, B)$ de avec $|B|$ minimal.

    Supposons par l'absurde que $H$ ne soit pas connexe, alors il existe $C_1,
    \hdots, C_p$ ($p \geqslant 2$) composantes connexes de $H$. Notons $n_i$
    et $m_i$ le nombres de sommets et d'aretes de $C_i$. Alors comme les $C_i$
    sont connexes et acycliques, ce sont des arbres, donc possèdent $m_i = n_i
    - 1$ aretes d'après \ref{itm:Q9}.

    Donc $|B| = \sum_{k=1}^{p} m_k = \sum_{k=1}^{p} n_i - 1 = n - p < n - 1$
    $\rightarrow$ absurde car $|B| \geqslant n - 1$. Donc finalement $H$ est
    connexe et acyclique, donc est un arbre, donc possède $|B| = n - 1$ aretes.

    Donc finalement $H = G$ et donc $G$ est un arbre, d'où $(iii) \implies (i)$
    
    \item \begin{minted}[breaklines]{ocaml}
let rec representant partition sommet =
    if partition.(sommet) < 0 then sommet
    else representant partition partition.(sommet)
    \end{minted}

    \item \begin{minted}[breaklines]{ocaml}
let union partition sommet1 sommet2 =
    let h_sommet1 = - partition.(sommet1) - 1 in
    let h_sommet2 = - partition.(sommet2) - 1 in
    if h_sommet1 > h_sommet2 then
        partition.(sommet2) <- sommet1
    else if h_sommet1 = h_sommet2 then
        begin
        partition.(sommet1) <- sommet2 ;
        partition.(sommet2) <- partition.(sommet2) - 1 
        end
    else
        partition.(sommet1) <- sommet2 ;;
    \end{minted}

    \item \label{itm:Q13} Montrons le résultat sur $k$ le nombre de réunions 
    réalisés:
    
    Notons $H_k$ : "Après $k$ réunions, si $s$ représentant de 
    $X \in \mathscr{P}$, alors $|X| \geq 2^{h(s)}$"
    \begin{itemize}
        \item Si $k = 0$, alors $\mathscr{P} = \mathscr{P}^{(0)}_n$, donc
        si $X \in \mathscr{P}$, $X = \{s\}$ et $h(s) = 0$ ainsi 
        $|X| = 1 \geq 2^{h(s)}$, d 'où ${H}_0$

        \item Soit $k \in \bb{N}^*$, supposons $H_{k - 1}$, montrons $H_k$.
        
        Si $\mathscr{P} = \{X_1, \hdots, X_p\}$ une partition ayant subi $k-1$
        réunions depuis $\mathscr{P}^{(0)}_n$.

        Soit $\mathscr{P}'$ ayant subi une réunion depuis $\mathscr{P}$.
        Notons $\mathscr{P}' = \{X_1', \hdots, X_{p - 1}'\}$.

        \vspace{1em}

        Soit $i \in \llbracket 0, p \llbracket$, alors si $X_i'$ n'a pas subi
        de réunion, alors il existe $j$ tel que $X_i' = X_j$ et donc si $s$ est 
        un représentant de $X_i'$, alors $h'(s) = h(s)$ et donc $|X_i'| = |X_j|
        \geqslant 2^{h(s)} = 2^{h'(s)}$.

        \vspace{1em}

        Sinon, $X_i'$ a subi une réunion et donc il existe $j \neq m$ tels que
        $X_i' = X_j \cup X_m$.

        Notons $s, s_j, s_m$ les représentants respectifs de $X_i', X_j, X_m$.
        Lors de la réunion, la hauteur de $s$ ne peut qu'augmenter de $1$, donc
        $h'(s) \leqslant 1 + h(s_j)$ et $h'(s) \leqslant 1 + h(s_m)$.

        Donc, $\mathscr{P}$ étant une partition, $X_j \cap X_m = \varnothing$
        et donc, 
        \begin{align*}
            |X_i'| &= |X_j| + |X_m| \\
                   &\geqslant 2^{h(s_j)} + 2^{h(s_m)} \\
                   &\geqslant 2^{h(s) - 1} + 2^{h(s) - 1} \\
                   &= 2^{h(s)}
        \end{align*}

        D'où $H_k$, et donc le résultat.
    \end{itemize}

    \item Si $\mathscr{P}$ est une partition construite depuis 
    $\mathscr{P}^{(0)}_n$, alors dans le pire des cas, trouver le représentant
    $s$ de $x \in S_n$ se fait en $h(s)$ appels récursifs. Or d'après 
    \ref{itm:Q13}, $h(s) \leqslant \log_2 |X|$, et $|X| \leqslant n$ (dans le
    pire des cas c'est une égalité). Donc la complexité de représentant est
    $\mathcal{O}(\log_2 n$)

    La fonction union ne fait que des opérations élémentaires sur des 
    \verb|array| donc est en $\mathcal{O}(1)$

    \item Pour vérifier que $G$ est un arbre, on vérifie qu'il possède $n - 1$ 
    aretes et qu'il est connexe.
    
    \begin{minted}[breaklines]{ocaml}
let count_composantes_connexes  graphe =
    let n = Array.length graphe in
    let count =  ref 0 in
    let partition = Array.make n (-1) in
    for sommet1 = 0 to n - 1 do
        List.iter (fun sommet2 ->
                let representant_sommet1 = representant partition sommet1 in
                let representant_sommet2 = representant partition sommet2 in
                if representant_sommet1 <> representant_sommet2 then
                    union partition representant_sommet1 representant_sommet2)
                graphe.(sommet1)
    done;
    for sommet = 0 to n - 1 do
        if partition.(sommet) < 0
            then incr count
    done;
    !count ;;

let est_arbre graphe =
    let n = Array.length graphe in
    (nombre_aretes graphe = n - 1) && (count_composantes_connexes graphe = 1)
    \end{minted}

    \item Il correspond au chemin $1-2-5-4$

    \item L'algorithme ne termine pas toujours, en effet si $G = G_{3,2}$ et
    $\mathcal{T} = (\{0\}, \varnothing)$, alors si $s = 5$, il se peut que
    l'algorithme fasse le chemin $1-2-6-5$ en boucle, le choix étant aléatoire,
    et donc l'extrémité d'un tel chemin ne se trouvera jamais dans 
    $\mathcal{T}$.

    \item \begin{minted}[breaklines]{ocaml}
let marche_aleatoire adj parent sommet =
    let chemin = {debut = sommet ; fin = sommet ; 
    suivant = Array.make (Array.length adj) 0} in
    while parent.(chemin.fin) = -2 do
        begin
        let nombre_voisins = Array.length adj.(chemin.fin) in
        let indice_voisin_aleatoire = Random.int nombre_voisins in
        let voisin_aleatoire = adj.(chemin.fin).(indice_voisin_aleatoire) in
        chemin.suivant.(chemin.fin) <- voisin_aleatoire; (*si u est dans le cycle, 
           cette modification n'importe pas*)
        chemin.fin <- voisin_aleatoire;
        end
    done;
    chemin ;;
    \end{minted}

    \item \begin{minted}[breaklines]{ocaml}
let greffe parent chemin =
    let sommet = ref chemin.debut in
    while !sommet <> chemin.fin do (*chemin.fin etant dans T, on a pas à l'ajouté*)
        let suivant = chemin.suivant.(!sommet) in
        parent.(!sommet) <- suivant;
        sommet := suivant
    done;
    \end{minted}

    \item \begin{minted}[breaklines]{ocaml}
let wilson g r =
    let n = Array.length g in
    let adj = adjacence g in
    let parent = Array.make n (-2) in
    parent.(r) <- -1 ;
    for sommet = 0 to n - 1 do
        if parent.(sommet) < 0 then
            let chemin = marche_aleatoire adj parent sommet in
            greffe parent chemin
    done;
    parent ;;
    \end{minted}

    \item \image{\centering}{0.5}{q21}
    
    \item \image{\centering}{0.4}{q22}
    
    \item Si $s$ un sommet de $\mathcal{T}$, alors les coordonnées de $s$ 
    dans $G_{p, q}$ sont $(i = \lfloor s / p \rfloor, j = s \mod p)$. 

    Comme $G_{p, q}$ ne garde que les cases noires, les coordonnées 
    correspondantes dans $E_{p, q}$ sont $(2i, 2j)$. 
    
    Ainsi en fonction de la direction du domino dans la case noire $(2i, 2j)$,
    on obtient les coordonnées de $s'$, le père de $s$ :
    \begin{itemize}
        \item Si la direction est OUEST, alors $s' = ip + (j - 1)$
        \item Si la direction est NORD, alors $s' = (i + 1)p + j$
        \item Si la direction est SUD, alors $s' = (i- 1)p + j$
        \item Si la direction est EST, alors $s' = ip + (j + 1)$
    \end{itemize}

    \item \begin{minted}[breaklines]{ocaml}
let coord_noire sommet = 
    let i, j = sommet / p, sommet mod p in
    (i * 2, j * 2) 
    \end{minted}

    \item \begin{minted}[breaklines]{ocaml}
let sommet_direction sommet direction =
    let i, j = sommet / p, sommet mod p in
    match direction with
    |N -> if i >= (q - 1) then -1 else (i + 1) * p + j
    |S -> if i <= 0 then -1 else (i - 1) * p + j
    |W -> if j <= 0 then -1 else i * p + j - 1
    |E -> if j >= (p - 1) then -1 else i * p + j + 1       
    \end{minted}

    \item \begin{minted}[breaklines]{ocaml}
let phi pavage = 
    let parent = Array.make (p * q) (-2) in
    parent.(0) <- -1;
    for i = 0 to q - 1 do
        for j = 0 to p - 1 do
            let sommet = i * p + j in
            let x_pavage, y_pavage = coord_noire sommet in
            if x_pavage mod 2 = y_pavage mod 2
                && (x_pavage, y_pavage) <> (0, 0) then
                let pere = sommet_direction sommet 
                                  pavage.(x_pavage).(y_pavage) in
                parent.(sommet) <- pere
          done;
    done;
    parent ;;       
    \end{minted}
\end{question}

\end{document}