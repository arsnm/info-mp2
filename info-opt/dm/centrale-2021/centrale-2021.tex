\documentclass{article}

% PACKAGES
% essentials
\usepackage[french, english]{babel} % specific language formatting
\usepackage[T1]{fontenc} % characters encoding
\usepackage{caption} % captions for graphics, images, ...

% page formatting
\usepackage[a4paper, portrait, margin=20mm]{geometry} % define the page format
\usepackage[explicit]{titlesec} % formatting and styling of section titles ... 
% ... and headers
\usepackage{fancyhdr} % customize headers and footers
\usepackage{lastpage} % keeps track of the number of last page
\usepackage{graphicx} % image, figures and graphics insertion
\usepackage[ddmmyyyy]{datetime} % display the date and time of saving
\usepackage{adjustbox} % scaling, resizing graphics
\usepackage{color, soul} % text formatting (highlight, underline, ...)
\setulcolor{red}
\usepackage{enumitem, hyperref}

% math notation
\usepackage{amsmath, amsfonts, amssymb}
\usepackage{stmaryrd}

% code formatting
\usepackage[outputdir=.aux]{minted}
\usemintedstyle{emacs}

%------------------------------------------------------------------------------

% COMMANDS AND PERSONALIZATION

% personalized section style
\titleformat{\section}
{\Large\bfseries}
{\thesection}{1em}{\ul{#1}}

% personalized 'question' list (mainly for test sheet)
\newlist{question}{enumerate}{2}
\setlist[question, 1]
{label=\bfseries{Q\arabic{questioni}.},
leftmargin=5em,
resume
}
\setlist[question, 2]
{label=(\alph{questionii}),
leftmargin=\parindent
}

% math commands
\newcommand{\bb}[1]{\mathbb{#1}}

% framing text
\newcommand{\encadrer}[1]{\framebox{color=red
    \begin{minipage}{0.90\textwidth}
        #1
    \end{minipage}
}}

\renewcommand{\thesection}{\Roman{section}} % Roman numerals for sections
\renewcommand{\thesubsection}{\thesection.\Alph{subsection} -} % Alphabetical
% numerals for subsections
\setlength{\headheight}{12.5pt}

\graphicspath{ {./img/} } % define path to img 

% insert image command
\newcommand{\image}[3]{
    \begin{minipage}[t]{\linewidth}
        #1
        \adjustbox{valign=t}{
            \includegraphics[width=#2\linewidth]{#3}
        }
    \end{minipage}}

% page numbering
\pagestyle{fancy}
\fancyhf{}
\renewcommand{\headrulewidth}{0pt}
\fancyfoot[R]{\thepage/\pageref{LastPage}}

%------------------------------------------------------------------------------

% DOCUMENT
% document info
\makeatletter
\title{Centrale 2021}
\date{\today}
\newcommand{\matiere}{Option Informatique}
\newcommand{\classe}{MP*}
\author{Arsene MALLET}

% header
\fancypagestyle{firstpage}{
    \fancyhead[L]{\@author}
    \fancyhead[C]{\classe - \matiere}
    \fancyhead[R]{\@date}
}


\begin{document}

\thispagestyle{firstpage}

\begin{center}
    \huge\bfseries{\@title}
\end{center}

\section{Quelques Fonctions Auxiliaires}

\begin{question}
    \item \begin{minted}[breaklines]{ocaml}
let nombre_aretes g =
    let rec length_list l =
        match l with
        |[] -> 0
        |h::t -> 1 + length_list t 
    in
    let s = ref 0 in
    for i = 0 to Array.length(g) - 1 do
        s := !s + length_list g.(i)
    done;
    !s / 2 ;;
    \end{minted}

    \item \begin{minted}[breaklines]{ocaml}
let g_2_3 = [|
    [|3; 1|];
    [|4; 0; 2|];
    [|5; 1|];
    [|0; 4|];
    [|1; 3 ; 5|];
    [|2; 4|];
    |];;
    \end{minted}

    \item \begin{minted}[breaklines]{ocaml}
let adjacence g =
    let n = Array.length g in
    let adj = Array.make_matrix n n 0 in
    for i = 0 to n - 1 do
    List.iter (fun x -> adj.(i).(x) <- 1) g.(i) 
    done;
    adj ;; 
    \end{minted}
    
    \item \begin{minted}[breaklines]{ocaml}
let rang (p, q) (s, t) =
    let is, js = s / p, s mod p in
    let it, jt = t / p, t mod p in
    if it = is + 1 then
        (q - 1) * js + is
    else if jt = js + 1 then
        p * (q - 1) + (p - 1) * is + js
    else 
        failwith "Argument(s) invalide(s)" ;;
    \end{minted}

    \item \begin{minted}[breaklines]{ocaml}
let sommets (p, q) rg =
    if rg < p * (q - 1) then
        let is, js = rg mod (q - 1), rg / (q - 1) in
        let s = is * p + js in
        (s, s + p)
    else if rg < p * (q - 1) + q * (p - 1) then
        let shift = p * (q - 1) in
        let is, js = (rg - shift) mod (p - 1), (rg - shift) / (p - 1) in
        let s = js * (q + 1) + is in
        (s, s + 1)
    else failwith "Argument(s) invalide(s)" ;;
    \end{minted}

    \item \begin{minted}[breaklines]{ocaml}
let quadrillage p q =
    let graphe = Array.make (p * q) [] in
    let rec remplissage_graphe rg =
        if rg < p * (q - 1) + q * (p - 1) then
            let v1, v2 = sommets (p, q) rg in
            begin
            graphe.(v1) <- v2 :: graphe.(v1) ;
            graphe.(v2) <- v1 :: graphe.(v2);
            remplissage_graphe (rg + 1);
            end
    in
    remplissage_graphe 0 ;
    graphe ;; 
    \end{minted}
\end{question}

\section{Caractérisation des arbres}

\subsection{Propriétés sur les arbres}

\begin{question}
    \item Si $s, t \in S_n$, notons $s \ast t$, la relation "Il existe un
    chemin de $s$ à $t$". \\
    Montrons que $\ast$ est une relation d'équivalence sur $S_n$.
    
    \begin{itemize}
        \item Réflexivité : soit $s \in S_n$. Par convention, il existe 
        un chemin de $s$ à $s$. Donc $s \ast s$.

        \item Symétrie : soit $s, t \in S_n$, si $s \ast t$, alors il existe 
        un chemin $c = (s, s1, \hdots, s_{k - 1}, t)$. 
        Donc $\forall i \in \{0, \hdots, k - 1\}$, $\{s_i, s_{i + 1}\} \in A$
        donc $\{s_{i + 1}, s_i\} \in A$, donc le chemin 
        $c' = (t, s_{k - 1}, \hdots, s_1, s)$ existe et donc $t \ast s$

        \item Transitivité : soit $s, t, u \in S_n$ tels que si $s \ast t$ et
        $t \ast u$. Alors il existe $c1 = (s, s_1, \hdots, s_{k - 1}, t)$ \\
        et $c2 = (t, t_1, \hdots, t_{i - 1}, u)$. \\
        Donc en concaténant ces chemins, il existe 
        $c = (s, \hdots, s_{k - 1}, t, t_1, \dot, t_{k - 1}, u)$, 
        d'où $s \ast u$.
    \end{itemize}

    Ainsi comme les composantes connexes de $G$ sont les classes d'équivalence
    de $\ast$ et forment donc une partition de $S_n$.

    \item Soit $s, t$ deux sommets tels que $s \ast t$, en notant $len(c)$ la
    longueur d'un chemin $c$, \\ alors 
    $L = \{ len(c) \vert c \text{ chemin de $s$ à $t$}\}$ est une partie de
    $\bb{N}$, non-vide (puisque $s \ast t$), \\ donc il existe un plus petit 
    élément $k_0$ de $L$. D'où l'existence d'un plus court chemin de $s$ à $t$.

    Soit $c_0$ un plus court chemin de $s$ à $t$, notons le
    $c_0 = (s, s_1, \hdots, s_{k_0 - 1}, t)$. \\
    Si il existe $i \neq j$ tels que $s_i = s_j$ (on peut supposer sans perte
    de généralité que $i < j$) alors $c = (s, \hdots, s_i = s_j, s_{j + 1}, 
    \hdots, t)$, un chemin de longueur $k_0 - (j - i) < k_0$, ce qui contredit
    le caractère de plus court chemin de $c_0 \rightarrow$ absurde. Donc les
    sommets d'un plus court chemin sont distincts.

    \item \label{itm:Q9} Soit $k \in \llbracket 0, m \llbracket$, notons $s, t$ les extrémités
    de $a_k$.

    Supposons que $s$ et $t$ appartiennent à la même composante connexe de
    $G_k$, alors $s \ast_k t$. Ainsi en notant 
    $c_k = (s_0 = s, s_1, \hdots, s_{i - 1}, s_i = t)$ (avec $i > 1$), où les
    sommets de $c_k$ sont adjacents dans $G_k$, alors il existe un chemin $c$ 
    dans $G$ tel que $c = (s, \hdots, t, s)$ (car $a_k$ relie $s$ et $t$). Or
    $len(c) = i + 1 \geqslant 2$, donc il existe un cycle dans $G$. Or $G$ est 
    un arbre donc est acyclique $\rightarrow$ absurde !

    Ainsi les extrémités de $a_k$ appartiennent à deux composantes connexes
    différentes de $G_k$.

    En notant pout tout $i \in \llbracket 0, m \rrbracket$,
    $\varphi(i)$ le nombre de composantes connexes de $G_i$, alors
    $\phi(0) = n$ ($G_0$ est composé de $n$ sommets non reliés) et $\varphi(m) = 1$ ($G_m = G$ est
    un arbre, donc connexe).

    Donc si $k \in \llbracket 0, m \llbracket$ et $a_k = \{s, t\}$, alors
    d'après ce qu'on a fait juste avant, $s$ et $t$ sont dans deux composantes
    connexes différentes de $G_k$ et dans la même dans $G_{k + 1}$. Les
    composantes connexes étant disjointes, si $u \in S_n$ tel que $C_{u_k} \neq
    C_{s_k}$ et $C_{u_k} \neq C_{t_k}$, alors $C_{u_k} = C_{u_{k + 1}}$, 
    (les $C_{i_k}$ étant les composantes connexes de $G_k$ contenant $i$), d'où
    finalement $\varphi(k + 1) = \varphi(k) - 1$

    Par une récurrence immédiate, $\varphi(m) = \varphi(0) - m$, d'où 
    $m = n - 1$ et donc le résultat.

    \item D'après \ref{itm:Q9}, $(i) \implies (ii)$ et $(i) \implies (iii)$
    
\end{question}

\end{document}