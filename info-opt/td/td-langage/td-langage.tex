\documentclass{article}

% PACKAGES
% essentials
\usepackage[french, english]{babel} % specific language formatting
\usepackage[T1]{fontenc} % characters encoding
\usepackage{caption} % captions for graphics, images, ...

% page formatting
\usepackage[a4paper, portrait, margin=20mm]{geometry} % define the page format
\usepackage[explicit]{titlesec} % formatting and styling of section titles ... 
% ... and headers
\usepackage{fancyhdr} % customize headers and footers
\usepackage{lastpage} % keeps track of the number of last page
\usepackage{graphicx} % image, figures and graphics insertion
\usepackage[ddmmyyyy]{datetime} % display the date and time of saving
\usepackage{adjustbox} % scaling, resizing graphics
\usepackage{color, soul} % text formatting (highlight, underline, ...)
\setulcolor{red}
\usepackage{enumitem, hyperref}

% math notation
\usepackage{amsmath, amsfonts, amssymb}
\usepackage{stmaryrd}

% code formatting
\usepackage[outputdir=.aux]{minted}
\usemintedstyle{emacs}

%------------------------------------------------------------------------------

% COMMANDS AND PERSONALIZATION

% personalized section style
\titleformat{\section}
{\Large\bfseries}
{\thesection}{1em}{\ul{#1}}

% personalized 'question' list (mainly for test sheet)
\newlist{question}{enumerate}{2}
\setlist[question, 1]
{label=\bfseries{\arabic{questioni}.},
leftmargin=5em,
% resume
}
\setlist[question, 2]
{label=(\alph{questionii}),
leftmargin=\parindent
}

% math commands
\newcommand{\bb}[1]{\mathbb{#1}}

% framing text
\newcommand{\encadrer}[1]{\fbox{color=red
    \begin{minipage}{0.90\textwidth}
        #1
    \end{minipage}
}}

\renewcommand{\thesection}{\Roman{section}} % Roman numerals for sections
%\renewcommand{\thesubsection}{\thesection.\Alph{subsection} -} %  Alphabetical
% numerals for subsections
\setlength{\headheight}{12.5pt}

\graphicspath{ {./img/} } % define path to img 

% insert image command
\newcommand{\image}[3]{
    \begin{minipage}[t]{\linewidth}
        #1
        \adjustbox{valign=t}{
            \includegraphics[width=#2\linewidth]{#3}
        }
    \end{minipage}}

% page numbering
\pagestyle{fancy}
\fancyhf{}
\renewcommand{\headrulewidth}{0pt}
\fancyfoot[R]{\thepage/\pageref{LastPage}}

%------------------------------------------------------------------------------

% DOCUMENT
% document info
\makeatletter
\title{TD - Langages Rationnels}
\date{\today}
\newcommand{\matiere}{Informatique Option}
\newcommand{\classe}{MP*}
\author{Arsène MALLET}

% header
\fancypagestyle{firstpage}{
    \fancyhead[L]{\@author}
    \fancyhead[C]{\classe - \matiere}
    \fancyhead[R]{\@date}
}


\begin{document}

% personalized list style


\thispagestyle{firstpage}

\begin{center}
    \huge\bfseries{\@title}
\end{center}

\section{Règles Opératives}

\begin{question}
    \item Vrai.

    \item Faux : $e_1e_2 \in (e_1 | e_2)^*$ mais $e_1e_2 \notin e_1^* | e_2^*$

    \item Faux : $e_1e_1 \notin (e_1 e_2)^*$ mais $e_1e_1 \in e_1^* e_2^*$

    \item Vrai.
\end{question}

\section{Petites Questions}

\begin{question}

    \item $(c^*ac^*bcˆ* | c^*bc^*ac^*)$
    \item Ce langage corresond à $1(0|1)^*0^2 | 0$
    \item $(a(b | c) | b | c)^*(a | \varepsilon)$
    \item $(bc | b)^*a(bc | b)^*a(bc | b)^*$
    \item \begin{itemize}
        \item $L(\frac{1}{6}) = \varepsilon | 16^*$
        \item $L(\frac{1}{7}) = (142857)^*(\varepsilon | 1 | 14 | 142 | 1428 | 14285)$
    \end{itemize}
    \item TODO

\end{question}
 
\section{Distance de Hamming}

\begin{question}

    \item \begin{itemize}
        \item $\forall u, v \in \Sigma,\; d(u, v) \geqslant 0$
        \item $\forall u, v \in \Sigma,\; d(u, v) = 0 \implies u = v$
        \item $\forall u, v, w \in \Sigma,\; d(u, v) \leqslant d(u, w) + d(w, v)$
    \end{itemize}
Ainsi la distance de Hamming est bien une distance.

    \item
    $\mathcal{H}(L(0^*1^*)) = (0^*(1 | \varepsilon)0^*) | (0^*1^*(0 | \varepsilon)1^*)$

    \item 

    \item \begin{minted}{ocaml}
let rec voisinnage = function 
    |Vide -> Vide
    |Epsilon -> Epsilon
    |L(a) -> L(1 - a)
    |Union(e1, e2) -> Union(voisinnage e1, voisinnage e2)
    |Concat(e1, e2) -> Union(Concat(voisinnage e1, e2), Concat(e1, voisinnage e2))
    |Etoile(e) -> Concat(Etoile(e), Concat(voisinnage e, Etoile(e)))
    \end{minted}

\end{question}

\section{Hauteur d'étoile}

\begin{question}

    \item \begin{align*}
        h((ba^*b)^*) &= 1 + h(ba^*b) \\
                     &= 1 + max(h(b), h(a^*b)) \\
                     &= 1 + max(0, max(h(a^*), h(b))) \\
                     &= 1 + max(1 + h(a), 0) \\
                     &= 1 + 1 = 2 \\
    \end{align*}

    \item \begin{minted}{ocaml}
let rec hauteur = function
    |Vide |Epsilon |L(_) -> 0
    |Union(e1, e2) |Concat(e1, e2) -> max (hauteur e1) (hauteur e2)
    |Etoile(e) -> 1 + hauteur e
    \end{minted}

    \item Ce sont des langages finies
\end{question}
\end{document}
