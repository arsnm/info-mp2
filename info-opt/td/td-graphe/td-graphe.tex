\documentclass{article}

% pacakages
\usepackage[french]{babel}
\usepackage{caption}
\usepackage[T1]{fontenc}
\usepackage{amsmath, amsfonts, amssymb}
\usepackage{stmaryrd}
\usepackage{fancyhdr}
\usepackage{lastpage}
\usepackage{lipsum}
\usepackage{graphicx}
\usepackage[ddmmyyyy]{datetime}
\usepackage{adjustbox}
\usepackage[a4paper, portrait, margin=20mm]{geometry} % define the page format
\usepackage[explicit]{titlesec}
\usepackage{color, soul}
\setulcolor{red}

%reference for the items of 'enumerate'
\usepackage{enumitem, hyperref}
\makeatletter
\def\namedlabel#1#2{\begingroup
    #2%
    \def\@currentlabel{#2}%
    \phantomsection\label{#1}\endgroup
}
\makeatother

%personalized section style
\titleformat{\section}
{\Large\bfseries}
{\thesection}{1em}{\ul{#1}}

%code formatting
\usepackage{minted}
\usemintedstyle{manni}

%divers commands
\newcommand{\bb}[1]{\mathbb{#1}}
\newcommand{\encadrer}[1]{\fbox{color=red
    \begin{minipage}{0.90\textwidth}
        #1
    \end{minipage}
}}
\renewcommand{\thesection}{\Roman{section}} % Roman numerals for sections
\setlength{\headheight}{12.5pt}

\graphicspath{ {./img/} } % define path to img 
\newcommand{\image}[3]{ %command to insert image
    \begin{minipage}[t]{\linewidth}
        #1
        \adjustbox{valign=t}{
            \includegraphics[width=#2\linewidth]{#3}
        }
    \end{minipage}}

%page numeration
\pagestyle{fancy}
\fancyhf{}
\renewcommand{\headrulewidth}{0pt}
\fancyfoot[R]{\thepage/\pageref{LastPage}}

%document info
\makeatletter
\title{TD - Graphe}
\date{\today}
\newcommand{\matiere}{Informatique Option}
\newcommand{\classe}{MP\textsuperscript{*} }
\author{Arsène MALLET}

%header
\fancypagestyle{firstpage}{
    \fancyhead[L]{\@author}
    \fancyhead[C]{\classe - \matiere}
    \fancyhead[R]{\@date}
}


\begin{document}

\thispagestyle{firstpage}

\begin{center}
    \huge\bfseries{\@title}
\end{center}

\section{Petites Questions}

\section{Théorème de Mantel}

\begin{enumerate}
    \item Soit $(u, v) \in E$. Comme chaque sommet de $G - u - v$ n'est connecté qu'à $u$ ou $v$ (car sinon triangle), alors $$\deg(u) + \deg(v) \leq |V|$$
    \item En réécrivant la somme, on a l'égalité : $$ \sum_{v \in V} \deg(v)^2 = \sum_{(u, v) \in E} \deg(u) + deg(v) \leq |E||V| $$
    \item Comme $\sum_{v \in V} \deg(v) = 2|E|$, alors d'après Cauchy-Schwartz : 
    \begin{align*}
        \sum_{v \in V} \deg(v)^2 \geq \frac{(\sum_{v \in V} \deg(v) )^2}{|V|} & = \frac{4|E|^2}{|V|} \\
        \text{donc } \frac{4|E|^2}{|V|} & \leq |E||V| \\
        \text{d'où } |E| & \leq \frac{|V|^2}{4} \\
    \end{align*}
    \item Les graphes bipartis au nombres de sommets paire fonctionnent.
\end{enumerate}

\end{document}
