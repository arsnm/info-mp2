\documentclass{article}

% PACKAGES
% essentials
\usepackage[french, english]{babel} % specific language formatting
\usepackage[T1]{fontenc} % characters encoding
\usepackage{caption} % captions for graphics, images, ...

% page formatting
\usepackage[a4paper, portrait, margin=20mm]{geometry} % define the page format
\usepackage[explicit]{titlesec} % formatting and styling of section titles ... 
% ... and headers
\usepackage{fancyhdr} % customize headers and footers
\usepackage{lastpage} % keeps track of the number of last page
\usepackage{graphicx} % image, figures and graphics insertion
\usepackage[ddmmyyyy]{datetime} % display the date and time of saving
\usepackage{adjustbox} % scaling, resizing graphics
\usepackage{color, soul} % text formatting (highlight, underline, ...)
\setulcolor{red}
\usepackage{enumitem, hyperref}

% math notation
\usepackage{amsmath, amsfonts, amssymb}
\usepackage{stmaryrd}

% code formatting
\usepackage[outputdir=.aux]{minted}
\usemintedstyle{emacs}

%------------------------------------------------------------------------------

% COMMANDS AND PERSONALIZATION
% references for items in 'enumerate'
\makeatletter
\def\namedlabel#1#2{\begingroup
    #2%
    \def\@currentlabel{#2}%
    \phantomsection\label{#1}\endgroup
}
\makeatother

% personalized section style
\titleformat{\section}
{\Large\bfseries}
{\thesection}{1em}{\ul{#1}}

% personalized 'question' list (mainly for test)
\newlist{question}{enumerate}{2}
\setlist[question, 1]
{label=\bfseries{Question \arabic{questioni}.},
wide=\parindent,
resume
}
\setlist[question, 2]
{label=(\alph{questionii}),
leftmargin=\parindent
}

% math commands
\newcommand{\bb}[1]{\mathbb{#1}}

% framing text
\newcommand{\encadrer}[1]{\fbox{color=red
    \begin{minipage}{0.90\textwidth}
        #1
    \end{minipage}
}}

% \renewcommand{\thesection}{\Roman{section}} % Roman numerals for sections
\setlength{\headheight}{12.5pt}

\graphicspath{ {./img/} } % define path to img 

% insert image command
\newcommand{\image}[3]{
    \begin{minipage}[t]{\linewidth}
        #1
        \adjustbox{valign=t}{
            \includegraphics[width=#2\linewidth]{#3}
        }
    \end{minipage}}

% page numbering
\pagestyle{fancy}
\fancyhf{}
\renewcommand{\headrulewidth}{0pt}
\fancyfoot[R]{\thepage/\pageref{LastPage}}

%------------------------------------------------------------------------------

% DOCUMENT
% document info
\makeatletter
\title{X-ENS 2023 : Informatique B}
\date{\today}
\newcommand{\matiere}{Informatique Tronc Commun}
\newcommand{\classe}{MP\textsuperscript{*}}
\author{Arsène MALLET}

% header
\fancypagestyle{firstpage}{
    \fancyhead[L]{\@author}
    \fancyhead[C]{\classe - \matiere}
    \fancyhead[R]{\@date}
}


\begin{document}

\thispagestyle{firstpage}

\begin{center}
    \huge\bfseries{\@title}
\end{center}

\section{Différentiels par positions fixes}

\begin{question}
    \item
    \begin{minted}[breaklines, autogobble]{python3}
    def textes_egaux(texte1, texte2):
        for i in range(len(texte1)):
            if texte1[i] != texte2[i]:
                return False
        return True
    \end{minted}

    La complexité de \verb|textes_egaux| est en $O(n)$, où $n$ est la
    longueur des textes.

    \item
    \begin{minted}[breaklines, autogobble]{python3}
        def distance(texte1, texte2):
        count = 0
            for i in range(len(texte1)):
                if texte1[i] != texte2[i]:
                    count += 1
            return count
        \end{minted}
    
        La complexité de \verb|distance| est également en $O(n)$, où $n$ est la
        longueur des textes.
    \item
    \begin{minted}[breaklines, autogobble]{python3}
        def aucun_caractere_commun(texte1, texte2):
            occurence1, occurence2 = {}, {}
            for i in range(len(texte1)):
                if texte1[i] not in occurence:
                    occurence1[texte[i]] = 1
                else:
                    occurence1[texte1[i]] += 1
            for j in range(len(texte2)):
                if texte2[j] not in occurence  
    \end{minted}
\end{question}

\section{Différentiels sur des positions variables}

\begin{question}
    \item
\end{question}

\end{document}